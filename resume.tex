%________________________________________________________________________________________________
% @brief    LaTeX2e Resume
\documentclass[margin,line]{resume}
\addtolength{\oddsidemargin}{0.1in}
\usepackage{tgtermes}
\usepackage[T1]{fontenc}
%\renewcommand{\labelitemi}{\ }

%________________________________________________________________________________________________
\begin{document}
\name{\Large Steve Herrin}
\begin{resume}

    % Contact Information
    \section{\mysidestyle Contact\\Information}

    steve.herrin@gmail.com		\hfill www.github.com/steveherrin		\vspace{0mm}\\\vspace{0mm}%
    650-814-8865				\hfill www.linkedin.com/in/herrinsteve		\vspace{0mm}\\\vspace{-4.5mm}%
    San Jose, CA    				\hfill medium.com/@steveherrin		\vspace{0mm}\\\vspace{0mm}%

    %________________________________________________________________________________________________
    % Professional Experience
    \section{\mysidestyle Experience}

    \textbf{23andMe}, Mountain View, CA \hfill\vspace{1mm}\\\vspace{1mm}%
    \textsl{Engineering Manager: Data Services Team}\hfill March 2018 -- present\vspace{1mm}\\\vspace{-3mm}%
    \begin{itemize}
    \item Led successful effort to parallelize, improve performance, and package for reuse in different environments the production machine learning evaluation code
    \item Authored engineering's open source policy, streamlining the release of 2 (to date) open source packages
    \item Grew a data-focused engineering team from 2 engineers to 3 teams totaling 16 engineers, including a mix of leads, senior, and junior level individual contributors
    \end{itemize}

    \textbf{23andMe}, Mountain View, CA \hfill\vspace{1mm}\\\vspace{1mm}%
    \textsl{Software Engineering Individual Contributor (final title: Tech Lead)}\hfill January 2014 -- March 2018\\%\vspace{1mm}%
    \begin{itemize}
    \item Architected machine learning infrastructure and services, allowing training, evaluation, and easy promotion to production of models for dozens of users across multiple departments
    \item Created Genotyping Services, a Django webapp on AWS allowing external researchers to easily run genetic studies, increasing sales by over 2\% and leading to strategic data-sharing agreements
    \item Led the building of a web-based portal that allows internal and external researchers to dynamically query anonymized data stored in HBase for \textgreater 5 million customers
    %\item Migrated a 20 kLOC Django web application to the AWS cloud, upgrading the back-end from Python 2 to 3 and standardizing the front-end using Elm
    \item Designed and implemented a Python library that provides a unified API for accessing customer data across MySQL, HBase, and other data stores
    \item Built 3 generations of data pipelines with Celery, Luigi, and AWS Simple Workflow to run Python, C++, and R algorithms that impute, transform, and analyze petabytes of genetic data
    \end{itemize}

    %\textbf{Insight Data Science}, Mountain View, CA \hfill August 2013 -- December 2013\vspace{1mm}\\\vspace{1mm}%
    %\textsl{Postdoctoral Fellow}
    %\begin{itemize}
    %\item Developed Parksafely, a web app applying a heuristic algorithm to make parking recommendations, reducing bike theft risk by 40\% while requiring only 150~ft more walking on average. Used Flask, PostgreSQL/PostGIS, and Javascript.
    %\end{itemize}

    \textbf{SLAC National Accelerator Lab}, Menlo Park, CA \hfill May 2008 -- August 2013\vspace{1mm}\\\vspace{1mm}%
    \textsl{Research Associate}
    \begin{itemize}
    \item Applied machine learning and computer vision algorithms to improve detector energy resolution by 25\%
    \item Created a PHP logbook webapp with a MySQL backend for tracking work on the EXO-200 experiment
    \item Collected heterogeneous sensor data at 1 Hz from over 600 channels at a site with unreliable internet connectivity to protect \$10M of liquid xenon and for use in later analysis
    %\item Developed batch data pipelines using Python, C++, and shell scripts to routinely measure detector characteristics by processing TB of calibration data.
    \item Developed batch data pipelines using Python, C++, and shell scripts to process TB of calibration data
    \item Mentored 1--2 junior graduate students (at any given time) on lab, coding, and statistical technique
    \end{itemize}

    %\textbf{Stanford University}, Stanford, CA \hfill April 2008 -- December 2011 \vspace{0mm}\\
    %\phantom{\textbf{Stanford University}, Stanford, CA}\hfill (3 quarters)\vspace{1mm}\\\vspace{1mm}%
    %\textsl{Teaching Assistant}\hfill (3 quarters)%
    %\begin{itemize}
    %\item Supervised 8 to 12 undergraduate students in laboratory and classroom settings.
    %\item Communicated advanced physics concepts, including computational physics using MATLAB.
    %\end{itemize}

    \textbf{Rice University}, Houston, TX \hfill May 2005 -- May 2007 \vspace{1mm}\\\vspace{1mm}%
    and \textbf{University of Washington}, Seattle, WA \hfill June 2006 -- August 2006 \vspace{1mm}\\\vspace{1mm}%
    \textsl{Undergraduate Research Assistant}
    \begin{itemize}
    \item Implemented (in C++) and evaluated random forest and boosted decision tree algorithms that contributed to the discovery of single top quark production by Fermilab's D0 experiment
    %\item Investigated and benchmarked many different machine learning classification algorithms for their power to discriminate signals of supersymmetry from backgrounds.
    \end{itemize}

    %________________________________________________________________________________________________
    % Skills
    \section{\mysidestyle Skills}\vspace{0mm}%
    \textbf{Languages:} Python, Rust, Elm, C++, SQL, Shell Scripting, JavaScript/TypeScript, MATLAB/Octave, R
    \vspace{1mm}\\\vspace{0mm}%
    \textbf{Tools:} AWS, NumPy, SciPy, Celery, Luigi, Django, MySQL, PostgreSQL, Git, \LaTeX, HBase, Spark
    \vspace{1mm}\\\vspace{0mm}%
    \textbf{Other:} Machine Learning, Data Analysis, Statistics, Simulation, Sensors, Neutrino \& Particle Physics

    %________________________________________________________________________________________________
    % Projects
    \section{\mysidestyle Open Source}\vspace{0mm}%
    \textbf{SpookyOTP:} A lightweight Python implementation of TOTP/HOTP authentication
    \vspace{1mm}%\\\vspace{0mm}
    %\textbf{PBWT Rust:} A Rust implementation of the positional Burrows-Wheeler transform

    %________________________________________________________________________________________________
    % Education
    \section{\mysidestyle Education}

    \textbf{Stanford University}, Stanford, CA \hfill June 2013\vspace{-3mm}\\\vspace{-1mm}%
    \begin{itemize}
    \item Ph.D. (Physics)
    \end{itemize}\vspace{-1.5mm}

    \textbf{Rice University}, Houston, TX \hfill May 2007\vspace{-3mm}\\\vspace{-1mm}%
    \begin{itemize}
    \item B.S. (Physics)
    \end{itemize}\vspace{-1.5mm}

    %________________________________________________________________________________________________
    % Honors and Awards
    %\section{\mysidestyle Honors and\\Awards} 

    %\textbf{Heaps Prize} (Rice University) \hfill May 2007\vspace{-3mm}\\\vspace{-1mm}%
    %\begin{itemize}
    %    \item For excellence in undergraduate research
    %    \item Physics and Astronomy Department
    %\end{itemize}\vspace{-1.5mm}

    %\textbf{Bonner Book Award} (Rice University) \hfill May 2007\vspace{-3mm}\\\vspace{-1mm}%
    %\begin{itemize}
    %    \item For outstanding senior undergraduate student
    %    \item Physics and Astronomy Department
    %\end{itemize}\vspace{-1.5mm}

    %________________________________________________________________________________________________
    % Publications
    %\section{\mysidestyle Selected Publications}

    %J.B.~Albert, \textit{et al.} (EXO Collaboration),\hfill April 2016\\
    %``Cosmogenic backgrounds to \(0\nu\beta\beta\) in EXO-200'',\\
    %J.\ Cosmol.\ Astropart.\ Phys.\ {\bf 2016}, 04 029 (2016)

    %J.B.~Albert, \textit{et al.} (EXO Collaboration),\hfill January 2014\\
    %``An improved measurement of the \(2\nu\beta\beta\) half-life of Xe-136 with EXO-200'',\\
    %Phys.\ Rev.\ C {\bf 89}, 1 015502 (2014)

    %M.~Auger, \textit{et al.} (EXO Collaboration),\hfill July 2012\\
    %``Search for neutrinoless double-beta decay in \(^{136}\)Xe with EXO-200'',\\
    %Phys.\ Rev.\ Lett.\  {\bf 109}, 032505 (2012)

    %M.~Auger, \textit{et al.} (EXO Collaboration), \hfill May 2012\\
    %``The EXO-200 detector, part I: Detector design and construction'',\\
    %JINST {\bf7} P05010 (2012)

    %A.~Dobi, C.~Hall, S.~Herrin, \textit{et al.} (EXO Collaboration), \hfill December 2011\\
    %``A xenon gas purity monitor for EXO'',\\
    %Nucl.\ Instrum.\ Meth.\ A {\bf 659}, 215 (2011)

    % N. Ackerman, \textit{et al.} (EXO Collaboration),\hfill November 2011\\
    %``Observation of two-neutrino double-beta decay in \(^{136}\)Xe with EXO-200'',\\
    %Phys.\ Rev.\ Lett.\  {\bf 107}, 212501 (2011)

%________________________________________________________________________________________________
% Talks / Panels
%\section{\mysidestyle Selected Talks}

    %Panelist, ``Machine Learning''\\
    % XLDB Conference \hfill 1 May 2018

    %``Migrating Bioinformatics Pipelines to the Cloud''\\
    % Biological Data Science (Biodata) Conference \hfill 27 October 2016

    %``The Science of Predicting Elections''\\
    %SASS (SLAC Association for Student Seminars) \hfill 31 October 2012\\%
    %MASS (Meeting of Astrophysics Students at Stanford) \hfill 5 November 2012

    %``A Search for Neutrinoless Double Beta Decay with EXO-200''\\%
    %SLAC Experimental Seminar \hfill 12 June 2012

    %``The Turn-On of EXO-200''\\%
    %American Physical Society April Meeting \hfill 3 May 2011

    %``Machine Learning''\\
    %SASS (SLAC Association for Student Seminars)\hfill 10 June 2009

%________________________________________________________________________________________________
% References
%    \section{\mysidestyle References} 
%    {\sl Available on request.}

%________________________________________________________________________________________________
\end{resume}
\end{document}
