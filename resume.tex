%________________________________________________________________________________________________
% @brief    LaTeX2e Resume
\documentclass[margin,line]{resume}
\addtolength{\oddsidemargin}{0.1in}
\usepackage{tgtermes}
\usepackage[T1]{fontenc}
%\renewcommand{\labelitemi}{\ }

%________________________________________________________________________________________________
\begin{document}
\name{\Large Steve Herrin}
\begin{resume}

    % Contact Information
    \section{\mysidestyle Contact\\Information}

    jobs@steveherrin.com		\hfill www.github.com/steveherrin		\vspace{0mm}\\\vspace{0mm}%
    650-814-8865				\hfill www.linkedin.com/in/herrinsteve		\vspace{0mm}\\\vspace{-4.5mm}%
    San Jose, CA    				\hfill		\vspace{0mm}\\\vspace{0mm}%

%________________________________________________________________________________________________
    % Summary
    \section{\mysidestyle Summary}
    Engineering leader with over a decade of experience building and leading teams that use software, data, and machine learning to solve novel problems. Scientific (physics PhD) background with demonstrated adaptability to other fields like biotech.%________________________________________________________________________________________________
    % Professional Experience
    \section{\mysidestyle Experience}
    \textbf{Pathos AI}, Chicago, IL (Remote) \hfill\vspace{1mm}\hfill June 2022 -- present\\%\vspace{-1mm}
    \textsl{Vice President of Engineering}
    \begin{itemize}
    \item Grew an engineering team from zero to four engineers of varying seniority
    \item Built a data catalog and management system to search, utilize, understand, and audit access to thousands of datasets and analysis results
    \item Prototyped LLM RAG generative AI pipeline for scientific literature review
    \item Used GCP, Terraform, and Nextflow to automate data analysis and processing
    \end{itemize}

    \textbf{D2G Oncology}, Mountain View, CA \hfill\vspace{1mm}\hfill April 2021 -- May 2022\\%\vspace{-1mm}
    \textsl{Staff Software Engineer}
    \begin{itemize}
    \item Created \textit{in silico} simulations of PCR and DNA sequencing, used for oligo design, QC, and automated processing of several multiplexed sequencing runs per month
    \item Built a pipeline using pydantic to automate ETL and validation of Benchling LIMS data from an HTTP API to a PostgreSQL warehouse
    \item Developed a Python library exposing GraphQL and REST APIs for accessing and traversing a large knowledge graph of lab, sequencing, public, and analysis data
    \end{itemize}

    \textbf{23andMe}, Sunnyvale, CA \hfill\vspace{1mm}\hfill January 2014 -- March 2021\\%\vspace{-1mm}
    \textsl{Engineering Management (\(\sim\)2 years; title: engineering manager)}
    \begin{itemize}
    \item Created and grew 3 machine learning and data -focused engineering teams totaling 16 engineers, including a mix of leads and individual contributors
    \item Led committee of engineering leads to standardize interviewing guidelines and open source release processes, subsequently adopted organization-wide
    \end{itemize}
    \textsl{Engineering Individual Contributor (\(\sim\)5 years; final title: sr.\ tech lead engineer)}
    \begin{itemize}
    %\item Built Django and HBase platform for internal and external researchers to dynamically query \(k\)-anonymized data for \textgreater 10 million customers
    \item Architected containerized (Docker, AWS ECS) machine learning systems to ensure quality and reproducibility of models in a regulated medical device setting
    \item Designed and implemented a library with a unified API for accessing data \& metadata across application-specific data stores, eventually used for all customer content
    \item Created web portal for external researchers to recruit for genomic studies and receive data back, increasing sales by 2\% and producing strategic data-sharing agreements
    %\item Migrated a 20 kLOC Django web application to the AWS cloud, upgrading the back-end from Python 2 to 3 and standardizing the front-end using React and Typescript
    \item Developed and performed a maximum likelihood analysis combining private and public datasets to replace thousands of ineffective genotyping probes
    \item Built 3 generations of distributed data pipelines with Celery, Luigi, and AWS to run Python, C++, and R algorithms operating on petabytes of genetic data
    %\item Automated genotype calling for SNPs and genes using a combination of unsupervised and supervised ML techniques
    \end{itemize}
    
    %\textbf{Insight Data Science}, Mountain View, CA \hfill August 2013 -- December 2013\vspace{1mm}\\\vspace{1mm}%
   % \textsl{Postdoctoral Fellow}
   % \begin{itemize}
   % \item Developed Parksafely, a web app applying a heuristic algorithm to make bike rack recommendations on a map, reducing bike theft risk by 40\% while requiring only 150~ft more walking on average. Used Flask, PostgreSQL/PostGIS, and Javascript.
   % \end{itemize}

    %\newpage

    \textbf{SLAC National Accelerator Lab}, Menlo Park, CA \hfill May 2008 -- August 2013\vspace{1mm}\\\vspace{1mm}%
    \textsl{Research Associate}
    \begin{itemize}
    \item Applied machine learning \& statistics to improve detector energy resolution by 25\%
    \item Used computer vision algorithms to repurpose the detector for 3D cosmic ray muon reconstruction, yielding a 10x reduction in cosmogenic background uncertainty
    \item Built, networked, and programmed PLC control systems with over 600 channels of heterogeneous sensor data at a site with unreliable internet connectivity, successfully protecting \$10M of liquid xenon
    \item Created a PHP logbook webapp with a MySQL backend for tracking lab work
    \item Developed batch data pipelines using Python, C++, and shell scripts to routinely measure detector characteristics by processing TB of calibration data.
    %\item Coordinated hardware and analysis software development with remote teams distributed around the world 
    %\item Mentored 1--2 junior graduate students (at any given time) on lab, coding, and statistical technique
    \end{itemize}

    %\textbf{Stanford University}, Stanford, CA \hfill April 2008 -- December 2011 \vspace{0mm}\\
    %\phantom{\textbf{Stanford University}, Stanford, CA}\hfill (3 quarters)\vspace{1mm}\\\vspace{1mm}%
    %\textsl{Teaching Assistant}\hfill (3 quarters)%
    %\begin{itemize}
    %\item Supervised 8 to 12 undergraduate students in laboratory and classroom settings
    %\item Communicated advanced physics concepts, including computational physics using MATLAB and Python
    %\end{itemize}

    %\textbf{Rice University}, Houston, TX \hfill May 2005 -- May 2007 \vspace{1mm}\\\vspace{1mm}%
    %and \textbf{University of Washington}, Seattle, WA \hfill June 2006 -- August 2006 \vspace{1mm}\\\vspace{1mm}%
    %\textsl{Undergraduate Research Assistant}
    %\begin{itemize}
    %\item Implemented (in C++) and evaluated random forest and boosted decision tree algorithms that contributed to the discovery of single top quark production by Fermilab's D0 experiment
    %\item Investigated and benchmarked many different machine learning classification algorithms for their power to discriminate signals of supersymmetry from backgrounds
    %\end{itemize}

    %________________________________________________________________________________________________
    % Skills
    \section{\mysidestyle Skills}\vspace{0mm}%
    \textbf{Languages:} Python, Rust, C, C++, SQL, Shell Scripting, Elm, JavaScript/TypeScript, R
    \vspace{1mm}\\\vspace{0mm}%
    \textbf{Tools:} AWS, GCP, NumPy, SciPy, Scikit-Learn, Mypy, Pydantic, FastAPI, Flask, Django, React, MySQL, PostgreSQL, Git, HBase, Spark, \LaTeX
    \vspace{1mm}\\\vspace{0mm}%
    \textbf{Other:} Machine Learning, Data Analysis, Bayesian/Frequentist Statistics, Simulation, CI/CD, Sensors, Analog \& Digital Electronics, Neutrino \& Particle Physics%, Radio (Amateur Extra License), Experienced Underground Miner

    %________________________________________________________________________________________________
    % Projects
    %\section{\mysidestyle Open Source}\vspace{0mm}%
    %\textbf{SpookyOTP:} A lightweight Python implementation of TOTP/HOTP authentication
   % \vspace{1mm}%\\\vspace{0mm}
    %\textbf{PBWT Rust:} A Rust implementation of the positional Burrows-Wheeler transform

    %________________________________________________________________________________________________
    % Education
    \section{\mysidestyle Education}
    \textbf{Insight Data Science}, Mountain View, CA \hfill December 2013\vspace{-3mm}\\\vspace{-1mm}%
    \begin{itemize}
    \item Postdoctoral Fellowship
    \end{itemize}\vspace{-1.5mm}

    \textbf{Stanford University}, Stanford, CA \hfill June 2013\vspace{-3mm}\\\vspace{-1mm}%
    \begin{itemize}
    \item Ph.D. (Physics)
    \end{itemize}\vspace{-1.5mm}

    \textbf{Rice University}, Houston, TX \hfill May 2007\vspace{-3mm}\\\vspace{-1mm}%
    \begin{itemize}
    \item B.S. (Physics)
    \end{itemize}\vspace{-1.5mm}

    %________________________________________________________________________________________________
    % Honors and Awards
    %\section{\mysidestyle Honors and\\Awards} 

    %\textbf{Heaps Prize} (Rice University) \hfill May 2007\vspace{-3mm}\\\vspace{-1mm}%
    %\begin{itemize}
    %    \item For excellence in undergraduate research
    %    \item Physics and Astronomy Department
    %\end{itemize}\vspace{-1.5mm}

    %\textbf{Bonner Book Award} (Rice University) \hfill May 2007\vspace{-3mm}\\\vspace{-1mm}%
    %\begin{itemize}
    %    \item For outstanding senior undergraduate student
    %    \item Physics and Astronomy Department
    %\end{itemize}\vspace{-1.5mm}

    %________________________________________________________________________________________________
    % Publications
    %\section{\mysidestyle Selected Publications}

    %J.B.~Albert, \textit{et al.} (EXO Collaboration),\hfill April 2016\\
    %``Cosmogenic backgrounds to \(0\nu\beta\beta\) in EXO-200'',\\
    %J.\ Cosmol.\ Astropart.\ Phys.\ {\bf 2016}, 04 029 (2016)

    %J.B.~Albert, \textit{et al.} (EXO Collaboration),\hfill January 2014\\
    %``An improved measurement of the \(2\nu\beta\beta\) half-life of Xe-136 with EXO-200'',\\
    %Phys.\ Rev.\ C {\bf 89}, 1 015502 (2014)

    %M.~Auger, \textit{et al.} (EXO Collaboration),\hfill July 2012\\
    %``Search for neutrinoless double-beta decay in \(^{136}\)Xe with EXO-200'',\\
    %Phys.\ Rev.\ Lett.\  {\bf 109}, 032505 (2012)

    %M.~Auger, \textit{et al.} (EXO Collaboration), \hfill May 2012\\
    %``The EXO-200 detector, part I: Detector design and construction'',\\
    %JINST {\bf7} P05010 (2012)

    %A.~Dobi, C.~Hall, S.~Herrin, \textit{et al.} (EXO Collaboration), \hfill December 2011\\
    %``A xenon gas purity monitor for EXO'',\\
    %Nucl.\ Instrum.\ Meth.\ A {\bf 659}, 215 (2011)

    % N. Ackerman, \textit{et al.} (EXO Collaboration),\hfill November 2011\\
    %``Observation of two-neutrino double-beta decay in \(^{136}\)Xe with EXO-200'',\\
    %Phys.\ Rev.\ Lett.\  {\bf 107}, 212501 (2011)

%________________________________________________________________________________________________
% Talks / Panels
%\section{\mysidestyle Selected Talks}

%    Panelist, ``Machine Learning''\\
%    XLDB Conference \hfill 1 May 2018

%    ``Migrating Bioinformatics Pipelines to the Cloud''\\
%    Biological Data Science (Biodata) Conference \hfill 27 October 2016

    %``The Science of Predicting Elections''\\
    %SASS (SLAC Association for Student Seminars) \hfill 31 October 2012\\%
    %MASS (Meeting of Astrophysics Students at Stanford) \hfill 5 November 2012

    %``A Search for Neutrinoless Double Beta Decay with EXO-200''\\%
    %SLAC Experimental Seminar \hfill 12 June 2012

    %``The Turn-On of EXO-200''\\%
    %American Physical Society April Meeting \hfill 3 May 2011

    %``Machine Learning''\\
    %SASS (SLAC Association for Student Seminars)\hfill 10 June 2009

%________________________________________________________________________________________________
% References
%    \section{\mysidestyle References} 
%    {\sl Available on request.}

%________________________________________________________________________________________________
\end{resume}
\end{document}
